% sample.tex
\documentclass[pdf,hpi,slideBW,nocolorBG,nototal]{prosper}
\usepackage{ngerman}
\usepackage[T1]{fontenc}
%\usepackage{fancyhdr}
%\usepackage[utf8]{inputenc}


\slideCaption{Accumulator | Softwaretechnik 2009 | Felgentreff, Haase, Stoff, Wollert, Xylander | \today }

\title{Accumulator}
\subtitle{More Than a Simple Feed-Reader}
\author{Xylander, Wollert, Stoff, Haase, Felgentreff}
\email{\{thomas.stoff, oliver.xylander, konstantin.haase, tim.felgentreff, johannes.wollert\}@student.hpi.uni-potsdam.de}
\institution{
	Softwaretechnik I \\
	SS 2009 \\
        Hasso-Plattner-Institut \\
        Universit�t Potsdam\\
}

\Logo{\thepage}

\begin{document}

\maketitle

%\begin{slide}{Introduction}
%  \begin{itemize}
%  \item Common Feed-Readers have shortcomings
%    \begin{itemize}
%    \item RSS-Feeds are often broken
%    \item Readers have limited input options
%    \item Feeds are only presented as they are
%    \end{itemize}
%  \item The Accumulator was meant to solve some of these
%    \begin{itemize}
%    \item Filters can fix some feed's problems
%    \item We can read more than the common feed formats
%    \item Feeds can be tagged and shared
%    \end{itemize}
%  \end{itemize}
%\end{slide}
%\begin{slide}{Demo}
%  nice picture
%\end{slide}
%\begin{slide}{Agenda}
%  \begin{itemize}
%  \item Goals 
%  \item XP
%  \item System
%  \end{itemize}
%\end{slide}
%\begin{slide}{Goals}
%  \begin{itemize}
%    \item Read from different sources (RSS, Atom, IMAP, Monticello, Git, \ldots)
%    \item Filter sources to ease focus on the things that matter
%    \item Be extensible: allow easy integration of new sources and filters
%    \item Be open: do not force users into the web-frontend, offer RSS output
%  \end{itemize}
%\end{slide}
%\begin{slide}{Development Process}
%  Xtreme Programming
%  \begin{itemize}
%    \item User stories
%    \item Pair programming
%    \item Stand-up meetings
%    \item TDD
%    \item Continouus integration
%  \end{itemize}
%\end{slide}
\begin{slide}{Development Process (contd.)}
  \begin{itemize}
    \item Pair programming
      \begin{itemize}
	\item Working in two's or three's
	\item Two or three teams working alongside 
	  one another on different parts of the system
	\item Working in one room, work-by-smoke-break
      \end{itemize}
  \end{itemize}
\end{slide}
\end{document}
