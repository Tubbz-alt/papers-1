\section{Design Process}
\subsection{Design Goals}
In order to achieve a pleasant and cleanly written FrozenBubble clone, we specified certain design goals.
A goal we set for ourselves very early was themeability, e.g. the ability to modify and 
adjust certain visual and logical aspects of the game (for example ball images or 
cannon positions). This demands an efficient and maintainable theme management 
system, as lots of pictures and data need to be processed.
As previously stated, another important aspect of our clone is ''realistic'' 
collision between objects, which came with certain performance considerations.
Lastly, we wanted our game to support multiplayer functionality. Given the time 
needed to implement networking, we decided to focus on a local multiplayer modes. 

\subsection{Design Problems and Solutions}
\subsubsection{Slow Disk I/O.}
\label{sec:theme}
%
Different from the original, we wanted to decouple the game from its representation 
to the point that we could allow different themes to change the positions of the 
game objects for different images and backgrounds. We decided to use png images to allow 
great visual flexibility and XML files to define the theme's positions. As disk I/O
is expensive, we needed a way to avoid it as much as possible, without sacrificing 
flexibility. Thus, lazy initialization for our images holding class was not an option. 
We created a class solely responsible for loading images and use the flyweight 
pattern to avoid reloading a theme we already loaded.
%
\subsubsection{Collision Between Balls.}
\label{sec:collision}
When a ball is shot into the playfield, it has to interact with the plafield 
and other objects in its path. Simply looping over all those objects and 
testing for collision is in O(n) and results in slower 
movement of the ball and a general loss in responsiveness. A quick solution 
would have been to keep two lists of all objects sorted by their x and by their 
y axis, and only testing those items closest to the current ball in both lists.
Now O(n) would only be the worst case. 
It occured to us that using rasterization or using precalculated vectors similar 
to ray-tracing techniques could get us a constant complexity. We decided 
to implement rasterization, because it was a faster solution and allowed straightforward
 optimization.
%
\subsubsection{Falling Balls.}
\label{sec:garbage}
When balls can collide and stick to each other, having them fall down 
if three or more of the same kind touch is trivial. However, balls 
hanging off balls that fall may have to fall as well. We considered 
two ways of checking for such conditions. We first planned to regard balls in the 
playfield as a network of nodes and use a graph search algorithm to determine 
whether or not a given ball has a connection to the top of the playfield via 
others. This proved difficult to implement and hurting encapsulation. Thus 
we applied a metaphor of a garbage collector (GC) to the problem, where balls
ball that are not connected to the top are collected. Performance was no pressing 
concern here, as the number of balls in one playfield is relatively limited, 
and the algorithm runs at a moment where the user has just shot, so taking a second 
for the calculation would not hurt the game flow.

\subsubsection{Simple Multiplayer Mode.}
\label{sec:playfield}
Several problems occured when we implemented a simple local multiplayer mode. In this chapter, 
we will discuss the most important design decissions.
%
  \paragraph{Players need to be informed of critical actions happening in other players' playfields}
  ~\\
    Direct communication between playfields would be the most simple solution like and 
    could be implemented easily. Additionally only the most rudimentary communication is needed,
    so there would be very few overhead. But this is not an object-oriented solution and playfields 
    would need to be enumerated, making the code hardly maintainable and badly extensible.\\
    In contrast, delegating communication through a mediator makes very maintainable and
    expandable code. Other than the previous proposal, this is indeed an object-oriented 
    take on the problem. However, there is more than twice as much communication happening 
    compared to the previous, simplified solution. Due to the obvious disadvantages of 
    hardcoding different playfields, we chose the latter proposal.

  \paragraph{Different playfields have to be controlled by different players.
             And, of course, different playfields have to exist in the first place}
  ~\\
    Our first idea we came up with was enabling the generic playfield to manage all needed behaviour
    on itself. That way, no new code needs to be introduced, rendering this scheme very simple.
    On the other hand, the playfield might have to be extended for all purposes to work properly. 
    This might lead to a very huge class indeed, affecting the readability.\\
    Introducing an Abstract Factory for playfields was another idea, thus implementing several 
    behavioural patterns. The obvious advantage of this proposal is, that Abstract Factories 
    are extensible and object-oriented. The downside of it would be, that this is a very powerful 
    pattern and would probably be overindulgence because we just need two different playfields 
    (single- and multiplayer behaviour). Because of that, we chose our initial idea of keeping 
    things simple.

  \paragraph{Key events need to be redirected to the corresponding playfield}
  ~\\
    The most easy-to-code version would be the introduction of a major, keyhandling class that
    knows what to invoke when a certain event takes place. Obviously, this has several downsides,
    as there would be strong coherence between objects and no encapsulation at all.\\
    As such a way of handling events is not maintainable and hardly extensible, we chose to catch
    key events through a class but unlike the previous solution, more object-oriented and maintainable
    patterns would be applied.\\
    An option is the Observer pattern where every object registers itself for certain keystrokes at
    a dispatch-object. That way, only objects that actually handle a specific event are informed.
    However, a lot of communication is happening that way.\\
    Another pattern that can be considered is the Chain of Responsibility, passing the event
    in question to other objects, when it cannot be handled by oneself. It is a very intuitive
    and object-oriented way of solving our problem, but has the same disadvantages of the 
    Observer pattern. In fact, the overhead problem is even more severe.\\
    As the complexity of a Chain of Responsibility is a lot less than that of an Observer pattern,
    we chose the latter scheme of handling keystrokes. Our problem just is not complex enough
    to justify a pattern as powerful as the Observer.
%

