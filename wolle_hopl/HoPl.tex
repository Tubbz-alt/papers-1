\documentclass{llncs}
\usepackage{makeidx} % allows for indexgeneration
\usepackage[pdftex]{graphicx} % PNGs
\usepackage{amsmath, amssymb} % algebra
\usepackage[latin1]{inputenc}
\usepackage[T1]{fontenc}
\usepackage[procnames]{listings} % for sourcecode
\usepackage{graphviz} % graphs
\usepackage{array,multirow} % tables
\usepackage{afterpage} % figures
\usepackage{float} % figures

\lstset{%
basicstyle=\small,
frame=single,
sensitive=true,
keywordsprefix=P,
keywords={END,W,W1},
keywordstyle=\bfseries,
identifierstyle=\ttfamily,
procnamestyle=\bfseries,
procnamekeys={P},
literate={<}{$<$}1 {>}{$>$}1 {=}{$=$}1 {<=}{$\leq$}1 {>=}{$\geq$}1 {=>}{{$\Rightarrow$}}1 {->}{{$\rightarrow$}}1
}

\restylefloat{figure}
\begin{document}
\pagestyle{headings} % switches on printing of running heads
\mainmatter % start of the contributions
\title{Erlang}
\subtitle{A concurrency-oriented Programming Language}
\titlerunning{Erlang} % abbreviated title (for running head)
\author{Johannes Wollert}
\date{\today}
\authorrunning{Johannes Wollert} % abbreviated author list (for running head)
\tocauthor{Johannes Wollert (Hasso-Plattner-Institute)}
\institute{History of Programming Languages, Software Architecture Group, Hasso-Plattner-Institut, Universit�t Potsdam, D-14482 Potsdam, Germany,\\
\email{johannes.wollert@student.hpi.uni-potsdam.de}}

\maketitle
\begin{abstract}
	In present times, concurrency is a matter of ever-growing importance, not only due to the movement towards multi-processor systems but also because of the inherently concurrent processes in our world. There are several models discribing possible solutions for the complex problems involved when handling largely concurrent and distributed systems such as scheduling, ressource management or communication between the multiple running processes or threads. \\
	The ''Erlang'' Programming Language implements an easy-to-use, yet powerful concept to tackle these challanges, providing an intuitive framework to build upon. 
\end{abstract}

\section{Introduction}
	Initially, Erlang was built for a single purpose alone, to significantly improve the development of telephony applications for Ericsson. To produce a language capable of manageing such systems, a large number of requirements need to be met. Of course, concurrency is one of them, but so is the ability to improve the system without even stopping it, a large amount of fault-tolerance and being able to control great numbers of processes, concurrently running in a distributed fashion. Also, telephony services need to be available all the time and because of that, the language must enable the designer to develop highly robust systems running ''forever''. \\
	Equipped with those clear goals, the idea started forming in 1985. In the following sections, we will outline the historical actions that lead to the formation of the Erlang language as we know it today. Later on, the language itself will be presented, covering syntax, types and paradigms.
	
\section{History}
  \subsection{The Need for Something new}
  	From 1974 until the end of the 1980's, the product generating most of Ericssons Profit was AXE, a telephony exchange written in PLEX, a proprietary language of Ericsson's used to specificly program the AXE hardware. It could not be used for any other devices. As this is not a worthwhile solution at all, Ericsson started the Computer Science Lab at 1986 with the intention of creating a platform for research concerning telecommunication to solve, what Bj�rne D�cker called ''Ericsson's software problem''. In order to achieve this, several changes or improvements over PLEX had to be made. 
  	\begin{enumerate}
  		\item 
  			The software had to be decoupled from the AXE exchange hardware. This was strongly tied to implementing a new and powerful model for processes, as PLEX' processes were a very peculiar hardware concept.
  		\item
  			The resulting language was to be very small, as small as possible while being able to describe large systems. Thus, training expenses could be reduced, a definite asset over PLEX for which Ericsson's employees had to be trained a long time to get productive in.
  		\item
  			Language-level primitives for concurrency were absolutely necessary in order to efficiently implement solutions for telephony exchange systems.
  	\end{enumerate}
  	
  \subsection{Joe Armstrong joins the Lab}
  	In early 1986, the software engineer Joe Armstrong arrived at the Computer Lab in order to work on the project. At first, research was moving towards finding an existing language capable of tackling the challanges. Armstrongs initial experiments dealed with implementing a rudimentary telephony program via Smalltalk. This however, proved to be a dead end at this time, because the SUN workstation Armstrong was working on turned out to be far to slow. ''So slow, that [he] used to take coffee brakes while it was garbage collecting''. At the same time, he was growing a graphical notation to visualize telephony exchange services. Upon showing this notation to a colleague, Roger Skagervall, he was first introduced to the Prolog declarative language, a large step for his strive towards a better way to program telephony.
  	
  \subsection{Expanding Prolog}
  	Realising the potential of Prolog, Armstrong started some experiments in adding concurrency to Prolog. Robert Virding, another Ericsson employee, joined Armstrong along the way. Virding was very experienced with logical programming languages, having implemented several extensions for concurrency in some. Through merging their distinctive views on concurrency and the way a language should be able to describe a concurrent system, the Prolog interpreter changed more and more into the direction of functional language, however, Virding and Armstrong never pointed out whether the language should be either functional or declerative or logical or imperative even. Instead, they took concepts from everywhere, fulfilling their requirements instead of strictly specifying the language itself. \\
  	At the same time, meaning early 1987, a first user group using the growing Prolog interpreter formed around a project named ACS/Dunder at the Lab. Their intention was to prototype a new telephony system and were looking for a simple language to fasten their pace. The interpreter, tagged Erlang by Bj�rne D�cker after Danish mathematician Agner Krarup Erlang, seemed perfect for that purpose. This provided necessary feedback for Armstrong and Virding, direct user input greatly stimulated the ongoing advancements of Erlang. \\
  	The 2-men-team was able to implement a large number of todays Erlang concepts during that time, primarily primitives for concurrency like send and receive. Those very important and crucial concepts were inspired mainly by the concurrent programming languages Modula and Chill, for example receiving messages is similar to a mailbox, just as it is in Chill. Yet some things remained untouched, partly because Prolog shielded them from those problems, particularly garbage collection, partly because Erlang was still considered a prototype language at Ericsson and therefor a very efficient, meaning ''soft'' realt-time-capable language was not necessary. While backtracking is a very useful feature in Prolog it actually hindered Erlangs speed due to its ''non-deterministic'' and therefor time-robbing approach. Another feature still missing in Erlang at that time was distribution, which a definitive requirement for a language with such aspirations as Erlang.
  	
  \subsection{ACS/Dunder Report}
  	Eventually, in December 1989 the ACS/Dunder project was finished and gave out their results and conclusions in an internal Ericsson paper. The most important results include:
  	\begin{enumerate}
  		\item
  			An average productivity increase per developer by factor 8 over PLEX, although it was later decreased to 3 to make it more credible.
  		\item
  			The time to implement a feature in Erlang was approximately 9 times shorter than to do the corresponding thing in PLEX.
  		\item
  			Erlang must be at least 40 times faster to meet the requirements of telephony applications. This was later corrected to a factor of 280.
  	\end{enumerate}
  	
  \subsection{Erlang needs a Virtual Machine}
  	The latter result being the only ''concerning'' result, Armstrong and Virding started actively thinking about the efficiency of Erlang for the first time. Efficiency in this context, however, would be largely hindered if Erlang continued to be an extended Prolog interpreter, so the team moved decoupling Erlang from Prolog and its virtual machine (VM). Thus, in 1990 Virding and Armstrong implemented a first VM (called the JAM) running byte-code, interpreting data compiled by a compiler written in Erlang itself which itself got bootstrapped through a Prolog emulator. This was, of course, far from the efficiency desired by Virding and Armstrong so work started to implement the emulator in C. However, both had very little experience with C. Luckily, Mike Williams started to gain interest in the project and decided to throw away all the emulator code written so far and started implementing a whole new emulator, making the overall Erlang implementation over 70 times faster in the end. \\
  	An important change occuring at the same time was the inclusion of an own parser. Gruadually, Erlangs syntax began to change, partly moving away from the infix operators of Prolog, resulting in the more ''functional feeling'' of today's Erlang.
  	
  \subsection{The Collapse of AXE-N and its Benefits}
	\begin{itemize}
  		\item[.]
  			\textit{In December 1995, a large project at Ellemtel, called AXE-N, collapsed.
This was the single most important event in the history of
Erlang.}
	\end{itemize} 
	During the course of 1990 to 1993, two new members, Claes Wikstr�m and Bogumil Hausman, joined the development team from inside the Lab, contributing to Erlang itself, but also holding courses for Ericsson employees, spreading Erlang inside the company. The prominent use of the language was prototyping, still. This changed in 1995 when, for the first time, Erlang was used as the main language to implement a major project at Ericsson. In its C++ implementation, the AXE-N exchange had collapsed under its own complexity and Erlang was used for the fresh restart with 60 developers involved. As a result the most important Open Telecom Platorm was built, which is documentation, beginner's course and vast library all in one. Today, the Erlang software distribution is called OTP. \\
  	This stunning success was made possible partly due to the implementation of a whole new VM by Bogumil Hausman which replaced the JAM entirely, making Erlang run approximately 10 times faster while also remarkably reducing the compiled code size.
  	
  \subsection{Erlang gets banned}
 	 \begin{itemize}
  		\item[.]
  			\textit{The selection of an implementation language implies a more
long-term commitment than the selection of a processor and
OS, due to the longer life cycle of implemented products.
Use of a proprietary language implies a continued effort to
maintain and further develop the support and the development
environment. It further implies that we cannot easily
benefit from, and find synergy with, the evolution following
the large scale deployment of globally used languages.}
	\end{itemize}
  	After several projects in the years after AXE-N, Erlang got banned at Ericsson for production use in 1998. Reasons involved were mostly because Ericsson was weary of maintaining its own proprietary languages instead of profiting from a large contributing community. This is however, what Joe Armstrong called the second most important event in the history of Erlang.
  	
  \subsection{Open Source Erlang}
  	After Erlang got open sourced in 1998 there were vast amounts of changes in Erlang, mainly a faster compiler produced by the HiPe Group and efficiency improvements in the Erlang Run-Time Environment. Today, some smaller companies work on Erlang and there is a large community with several mailing lists and thousands of contributing developers.
  
\section{The Language}
  \subsection{Paradigms}
  	Erlang is considered a multi-paradigm language due to its ''lack'' of focus on a single one, e.g. either functional or declarative or logical. As it was developed with certain needs in mind rather than a strong specification, it features several concepts of each though the main part is, in fact, functional. The term concurrency-oriented is often used when describing the language. This refers to the focus of the language to model largely concurrent systems.\\
  	Anyway, some design decisions have been made.
  	\subsubsection{Single-Assignment}
  		Values for variables can only be assigned once, then they are bound. Erlang is a single-assignment language, because Armstrong and Virding thought that this provided an easier way of debuging and error handling.
  	\subsubsection{Dynamic Typing}
  		Static Typing, e.g. executing type checks while compilation, is very hard in Erlang due to difficulties when type checking processes and Inter-Process-Communication. That way, however, type errors occur during run-time.
  		
  \subsection{Types}
  	Here, we will present the multiple number of types available in Erlang. All types are aggregated by the name ''term''.
  	\subsubsection{Number}
  		Integer and Float numbers are available in Erlang. Additionally, Integers can be initialized with $base \sharp value$.\\
  		For example: $4 \sharp 2$ refers to 4 to the power of 2.
  	\subsubsection{Atom}
  		These are similar to the Atoms in Prolog. They are literal constants defined by a name. They are used as tags and identifiers for functions, most of the time. Also, as there are no Booleans in Erlang, they are commonly used to replace those. $'$ is used to identify Atoms that start without a small character or consist of numerics and underscores. \\
  		For example: $'Hello', hello$
  	\subsubsection{Bit String}
  		Bit Strings are chunks of data of untyped memory. $<< Bit String >>$ identifies a Bit String. \\
  		For example: $<< "ABC" >>$
  	\subsubsection{Reference}
  		A unique term in an Erlang runtime system. \\
  		For example: $make_ref/0$
  	\subsubsection{Fun}
  		A Fun is a functional Object. It defines an anonymous function that is passed to other functions. \\
  		For example: $Fun1 = fun (X) -> X+1$ $end.$\\
  		$Fun1(2) = 3$
  	\subsubsection{Port}
  		A Port is used to provide basic communication facilities with the external world. Ports are opened using $open \_ port(PortName, PortSettings)$.
  	\subsubsection{Process Identifier (Pid)}
  		A Pid is used to uniquely identify an Erlang process. They are created whenever an Erlang process is spawned.
  	\subsubsection{Tuple}
  		A Tuple contains data with a fixed number of terms. Each term is called an element and can be accessed via its position in the Tuple. \\
  		For example: $A = \lbrace Atom, Number \rbrace$\\
  		$element(1,A) = Atom$
  	\subsubsection{List}
  		A List is a datacompound, containing a variable number of elements. These are similar to Prolog Lists.\\
  		For example: $[a,2,{c,4}]$

  \subsection{Pattern Matching}
  	In Erlang, variables are bound to values via the pattern matching process. The right hand pattern is being matched aiganst the left hand one and when the matching succeeds, the variables get bound. This concept was inherited from the ML language family. An example can be seen in the figure \ref{fig:pattern}\\
  	\begin{figure}
   		\begin{lstlisting}
1> X.
** 1: variable 'X' is unbound **
2 > X = 2.
2
3> X + 1.
3	
4> {X, Y} = {1, 2}.
** exception error: no match of right hand side value {1,2}
5> {X, Y} = {2, 3}.
{2,3}
6> Y.
3
   		\end{lstlisting}
	\caption{Pattern Matching in Erlang}
	\label{fig:pattern}
	\end{figure}
   
  \subsection{Module}
  	Erlang systems are divided into Modules. They consist of a sequence of attributes and functions, latter are terminated by a ''$.$''. A short example can be seen in figure \ref{fig:module}.
	\begin{figure}
   		\begin{lstlisting}
-module(m).          % module attribute
-export([fact/1]).   % module attribute

fact(N) when N>0 ->  % beginning of function declaration
    N * fact(N-1);   %  |
fact(0) ->           %  |
    1.               % end of function declaration
   		\end{lstlisting}
	\caption{A Module in Erlang}
	\label{fig:module}
	\end{figure}
  \subsection{Function}
  	Functions are parts of modules, implementing a certain amount of behaviour in it. They consist of a head and a body, whereas the head has a function name, an argument and an optional guard sequence. The body is simply a collection of expressions. This can be seen in figure \ref{fig:function}.
	\begin{figure}
   		\begin{lstlisting}
Name(Pattern1,...,PatternN) [when GuardSeq] ->
    Body1.
   		\end{lstlisting}
	\caption{A Function in Erlang}
	\label{fig:function}
	\end{figure}
	
  \subsection{Expression}
  	Expressions are various constructs, that can contain higher-level statements such as $if$ and $case$, patterns and functions. An example if-caluse can be seen in figure \ref{fig:if}.
  	
	\begin{figure}
   		\begin{lstlisting}
    if
        X>Y ->
            true;
        true -> % works as an 'else' branch
            false
    end
   		\end{lstlisting}
	\caption{An if-Clause in Erlang}
	\label{fig:if}
	\end{figure}

  \subsection{Language Properties}
	\subsubsection{Concurrency}
	\subsubsection{Distribution}
  	\subsubsection{Robustness}
  	\subsubsection{Hot Code Swapping}
  	\subsubsection{Portability}
\section{Popularity and Impact}
  \subsection{Major Erlang Applications}
  	\subsubsection{AXD301}
		In 1998 the largest systems ever to be built in Erlang was started at Ericsson, the AXD301. As the first major product using Erlang, it featured distribution. 
\section{Conclusions}

\end{document}