%&program=xelatex
%&encoding=UTF-8 Unicode
\documentclass[xelatex,notes=hide,14pt,sans,t]{beamer}

\usepackage{listings}
\usepackage{amssymb}
\usepackage{cite}
\bibliographystyle{plain}

% \usepackage{pgf}

\usepackage[quiet]{mathspec}
\setmathsfont[Set=Latin,Uppercase=Regular,Alternate=0]{Hoefler Text}

\usepackage{xunicode,xltxtra}

\setmainfont[Alternate=1,Ligatures={Common,Diphthong}]{Helvetica Neue}
\setsansfont[Mapping=tex-text]{Helvetica Neue}
\setmonofont{Inconsolata}

\mode<presentation>
{
  \useoutertheme{infolines}
  \useoutertheme{hci}
  \usefonttheme{hci}
  \usecolortheme{hci}
  \setbeamercovered{highly dynamic}
}

\lstset{language=CIL,
  basicstyle=\ttfamily,
  xleftmargin=15pt,
  frame=lines,
  numbers=left,
  numberstyle=\rmfamily\tiny,
  numbersep=3pt,                  % how far the line-numbers are from the code
  breaklines=true,                % sets automatic line breaking
  breakatwhitespace=true,        % sets if automatic breaks should only happen at whitespace
}

\title{FORTH}
\subtitle{A Stack Based Programming Language}
\author[Robert Pfeiffer]{Prof Robert Hirschfeld \\ Robert Pfeiffer} 
\institute[HOPL Seminar]{History of Programming Languages}
\date[SS 2009]{\today}

\begin{document}

\frame[plain]{\maketitle}
\tableofcontents

\section{Early History}
\frame{
  \frametitle{Beginnings}
  \framesubtitle{1950s–1968}
  \begin{itemize}
  \item developed by Charles R. "Chuck" Moore
    \note[item]{Studied Physics, then math}
    \note[item]{Chuck learned Lisp from John McCarthy at MIT}
    \note[item]{Jobs with satellites, }
    \note[item]{he worked for several Companies, programmed in Fortran, Cobol}
    \note[item]{did not want to program Cobol again, became Freelancer}
  
  \only<2>{\includegraphics{chuck0.jpg} \\ at the National Radio Astronomical Observatory \footnote{Picture taken from colorforth.com} \cite{chuckbio}}
  \only<3>{\includegraphics[width=5cm]{chuck1.jpg} \\ chuck moore today \footnote{Picture taken from Wikimedia Commons}}
   \end{itemize}

}
\frame{
  \frametitle{Beginnings}
  \framesubtitle{1950s–1968}
   \begin{itemize}
  \item developed by Charles R. "Chuck" Moore
 \item<2->Dynamic, "fourth"-generation language as opposed to C, Fortran, Algol
   \note[item]{Today, the term "fourth generation" refers to something different}
   \note[item]{declarative Programming possible, but not intended}
 \item<3-> First used in production to operate a telescope in an observatory in 1968
   \note[item]{in 1968 Forth looked much like today}
  \item<4-> Also first contact with other Programmers
 \item<5-> The IBM 1130 only allowed all-caps five-letter filenames, hence the name "FORTH"
 \end{itemize}
}

\frame{
  \frametitle{Design Goals}
  \begin{itemize}
  \item Simplicity \note[item]{Worse is better} \pause
  \item Expressive and customizable core language \pause
  \item Interactive development, incremental compilation \cite{forthvalues} \pause
    \note[item]{interactive development is THE importent feature, compiling whole programs was slooow back then}
  \item Small memory footprint \pause
    \note[item]{small deck of punch-cards, small ROM, low main memory}
  \item Close to the machine/low-level \pause
  \item Stand-Alone Forth system
  
     (Running without an OS)
      \note[item]{Portability was a non-goal}
  \end{itemize}
}
\section{The Language}

\subsection{Example}

\frame{
  \frametitle{Example Code}
  \framesubtitle{Try this in the GForth REPL}
  \lstinputlisting{example.forth}
  \note[item]{
    Hello World example  
    (3 + 4), almost good enough for Smalltalk
  }
}

\subsection{Syntax}

\frame{
  \frametitle{Syntax}
  \framesubtitle{}
  \begin{itemize}
  \item Forth Syntax looks like reverse polish notation (it is not)  \pause
  \note{not parsed}
  \item Sequence of tokens separated by whitespace  \pause
  \item either Literals or "Words"
  \end{itemize}  
}

\subsection{Semantics}

\frame{
  \frametitle{Semantics}
  \framesubtitle{}
      \note{Words are executed sequentially}
  A Literal is pushed onto the data stack.
  \pause
  
  A Word which implements a function pops its arguments off the data stack and pushes the result onto the data stack. \cite{181999}
  \pause
  
  \begin{alertblock}{Forth is not typed}
    The Data Stack contains untyped values. If there are values of the wrong type it may or may not work (undefined behaviour).
  \end{alertblock}
}

 
\frame{
  \frametitle{Semantics}
  Some Words are marked as "immediate". When an immediate word is encountered at compile time, it is executed by the compiler.
  \note[item]{immediate words can do their own parsing}
  \note[item]{Syntax is not formally specified, No complete parse tree}
}

\frame{
  \frametitle{Semantics}
  \framesubtitle{by example}
  A Forth Program contains a list of word definitions, called a dictionary.
  \note[item]{factoring: dividing stuff into words and giving meaningful names \\}
  \lstinputlisting[linerange=1-2]{dictionary.forth}
  \only<2->{the stack grows this way $\longrightarrow$}

  {\ttfamily
    \only<2>{\dots | n \\dup ( x -- x x )}
    \only<3>{\dots | n | n}
    \only<4>{\dots | n | n | 1\\> ( n m -- b )}
    \only<5>{\dots | n | $n > 1$}
    \only<6>{\dots | n | n}
    \only<7>{\dots | n | n | 1}
    \only<8>{\dots | n | $(n - 1)$}
    \only<9>{\dots | n | $fib(n - 1)$\\swap ( a b -- b a )}
    \only<10>{\dots | $fib(n - 1)$ | n}
    \only<11>{\dots | $fib(n - 1)$ | n | 2}
    \only<12>{\dots | $fib(n - 1)$ | $(n - 2)$}
    \only<13>{\dots | $fib(n - 1)$ | $fib(n - 2)$}
    \only<14>{\dots | $fib(n - 1) + fib(n - 2)$}}
 
  \note[item]{
    Parens enclose stack effect comments
    Words: dup, swap, drop, rot, over
  }
}

\frame{
  \frametitle{Semantics}
    \framesubtitle{by example}
    \lstinputlisting[linerange=4-7]{dictionary.forth}
  \only<2->{the stack grows this way $\longrightarrow$}
  
  {\ttfamily
    \only<2>{\dots | n}
    \only<3>{\dots | n | 0 | 1\\rot ( a b c -- b c a )}
    \only<4>{\dots | 0 | 1 | n}
    \only<5>{\dots | 0 | 1 | n | 1\\?do ( end start -- )}
    \only<6>{\dots | 0 | 1\\dup ( x -- x x )}
    \only<7>{\dots | 0 | 1 | 1\\rot ( a b c -- b c a )}
    \only<8>{\dots | 1 | 1 | 0}
    \only<9>{\dots | 1 | 1}
    \only<10>{\dots | 1 | 1 | 1}
    \only<11>{\dots | 1 | 1 | 1}
    \only<12>{\dots | 1 | 2}
    \only<13>{\dots | 1 | 2 | 2}
    \only<14>{\dots | 2 | 2 | 1}
    \only<15>{\dots | 2 | 3}
    \only<16>{repeat n times \dots}
    \only<17>{\dots | $fib(n - 1)$ | $fib(n)$}}
    \note[item]{
    this is better factored, less stack shuffling here
  }  
  
}

\subsection{Features}
  \frame{
    \frametitle{Algebraic Properties}
    \framesubtitle{Concatenative Language}
    \begin{itemize}
    \item No mutable variables 
          \note[item]{Side effects= return vals}

    \item Juxtaposition of words is function composition \pause
	   "{\ttfamily foo word1 word2}" means 
	   $ (word2 \circ word1)(foo) $
    \item Peephole optimization is easily possible
    \item Multiple return values
      \note[item]{call a continuation with multiple arguments,  no variables(functional), factoring}
    \end{itemize}
}

\section{The Forth VM}
\frame{
  \frametitle{Threaded Code}
  Words are compiled into the adresses of the code of their definitions.
  \pause
  The Forth VM iterates across an array of adresses (bytecode) \pause and jumps to each adress after saving the current location in the bytecode on the return stack.
  \pause
  Results:
    \begin{itemize}
    \item small executables
    \item simple compilation
    \item slower than C
    \item different calling convention
        \end{itemize}
      \note{can easily gauge run time,early binding}
}

\section{Demo}
\frame[plain]{
  \frametitle{Demo}
  \note{early binding, snake, code zeigen}
}

\section{Forth History Contiued}

\frame{
  \frametitle{Development}
  \framesubtitle{}
  \begin{itemize}
  \item Paper published in 1970 \pause
  \item Chuck Moore and Elizabeth Rather founded Forth INC. in 1971\cite{155369}\pause
  \item Forth Inc. marketed Forth compilers and systems \pause
  \item Ported to other platforms \pause
    \note[item]{MacForth on the Macintosh}
  \item Object-Oriented dialects \pause
    \note[item]{Neon, OOF-Systems}
  \end{itemize}
}

\frame{
  \frametitle{ANS Forth}
  \framesubtitle{Standardized in 1994}
  \begin{itemize}
  \item inofficial Specifications for Forth-79 and -83  \pause
  \item official Standard created by ANSI  \pause
  \item Established by Forth Systems Vendors  \pause
     \note[item]{Much Like Common Lisp Standard}
  \item Should provide that a portable subset of current Forth language Implementations  \pause
  \item Specified Data Word and Stack Size and Words that should be available 
    \note[item]{No interesting Features added}
  \end{itemize}
}

% \frame{
%   \frametitle{Development by Chuck Moore}
%   \framesubtitle{Chipdesign and ColorForth}
%   \begin{itemize}
%   \item after Forth INC: Moore simplifies Forth iteratively
%     \note[item]{Chuck did not like the ports to other operating systems or the new conecpts.
%       He refactored Forth and designed specialized hardware that could execute his programs faster.
%       He iteratively made it simpler. while others scaled up, he scaled down.}
%   \item Simper Forth Systems: Hardware implementations of Stack Machines
%     \note[item]{no more VMs}
%   \item ColorForth: new dialect
%       \note[item]{not Standards-Compatible}
%    \item Words have colors with different Semantics
%            \note[item]{or Fonts}
%    \item Complete System withHardware, Operating System and IDE
%    \end{itemize}
% }

\section{Forth Today}
\frame{
  \frametitle{Forth and You}
    \framesubtitle{Implementations}
  GForth (part of the GNU Project)
  
  Colorforth (by Chuck Moore)
  
  SwiftForth (by Forth INC.)
  
  RetroForth (available as a Dashboard widget)
  \pause
    \begin{block}{Learning Forth}
    \footnotesize
  Starting Forth \cite{1095598}
  
  Thinking Forth \cite{1616}
  
  The gForth Manual
 \end{block}
}

\frame{
  \frametitle{Usage}
  \framesubtitle{in Production}
   \begin{itemize}
   \item  Forth Interest Group (forth.org)
    \note[item]{The FIG also owns forth.org}
     \item  Forth-Gesellschaft e.V. (http://www.forth-ev.de/)
    \item Forth INC (forth.com) 
     \note[item]{Customers are Mercedes, Boeing, JPL, AT \&T,ATMEL}
    \item OpenFirmWare
     \note[item]{OFW is used in the OLPC}
  \end{itemize}

 }


\frame{
  \frametitle{Languages Influenced by Forth}
  
  Postscript
  
  Lisp, Smalltalk-80
  
  Java Bytecode, .Net CIL
}

\frame{
  \frametitle{Modern Stack Languages}
  Joy
  
  Factor 
  
  Cat
  
  False
  
  \includegraphics[width=3cm]{factor-logo.png}  \includegraphics[width=3cm]{cat-logo.jpg}% \includegraphics[width=4cm]{false-logo.gif}
}

\frame{
  \frametitle{Strengths}
      \framesubtitle{What is ist good for?}

  Embedded
   
  Real-Time Applications(no GC, no late binding)
  
  Extensible Compiler: DSLs
  
  Side effects are easy to reason about
}

\frame{
  \frametitle{Weaknesses}
        \framesubtitle{Why don't You use it?}

  No Module System
  
  No Typechecking ever
     
  Extensible Compiler
}

\frame[plain,allowframebreaks]{
  \frametitle{Sources}
%\tiny
\bibliography{forth}{}
}


\end{document}
