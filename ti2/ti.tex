\documentclass[a4paper]{article}
\usepackage{amsmath, amssymb}
\usepackage[ngerman]{babel}
\usepackage[utf8x]{inputenc}
\usepackage[T1]{fontenc}
\usepackage{listings}
\author{Tim Felgentreff, 738147}
\title{TI2}
\date{\today}

\setlength{\hoffset}{-1in}
\setlength{\voffset}{-1in}
\setlength{\topmargin}{0pt}
\setlength{\oddsidemargin}{0pt}
\setlength{\headheight}{0pt}
\setlength{\headsep}{0pt}
\setlength{\textheight}{842pt}
\setlength{\textwidth}{595pt}
\setlength{\marginparsep}{0pt}
\setlength{\marginparwidth}{0pt}
\setlength{\footskip}{0pt}


\begin{document}
\footnotesize
\noindent\fbox{\begin{minipage}{200pt}
  {\bf Rice und Diagonalisierung}  sind immer negative Beweise\\
  {\bf Rice} $M_{umgeformt}=\{i|\varphi_i\in P\}\quad \emptyset\neq P\subset\mathcal{R}\Rightarrow M$ ist unentscheidbar\\
  {\bf Curch} alle intuitiv lösbaren Probs sind turing-lösbar\\
  {\bf Kleene-Normalform} alles berechenbare brauch max. 1 while $\equiv h(x) = g(x,\mu f(x))$, $f,g$ sind $\mathcal{PR}$\\
  {\bf Cook} $SAT \in \mathcal{NPC}$
\end{minipage}}

\noindent\fbox{\begin{minipage}{200pt}
  {\bf H} $\{<i,j>|j\in domain(\varphi_i)\}$\\
  {\bf S} $\{i|\varphi_i(i) is definiert\}$\\
  {\bf Pr:} $h=Pr[f,g]$, Stelligkeit:$f+1=h=g-1$, Bsp:$h(x,y+1)=\big\{\binom{Pr^1_1(x)\text{ wenn y=0}}{Pr(x,y,h(x,y))}$\\
  {\bf Turing-mächtig:}  $Q$-logic, Registermaschine, $\lambda$-Kalkül, $\mu$-rekursiv\\
  {\bf NP} NTM kann in $\mathcal{P}$ lösen\\
  {\bf NPC} in $\mathcal{NP}$ und $\mathcal{NP}$-hart\\
  {\bf \boldmath$\lambda$-Kalkül}  Church-Numerals und von hinten anwenden\\
  {\bf Fixpunkt-Combinator}  $F t \equiv t (F t)$,z.B.:$\mathsf{Y}$\\
  {\bf Church-Numerals}  $\overline{1}=\lambda f.\lambda x.f x$, $\overline{2} = \lambda f. \lambda x f^2 x$ bzw. $\lambda f.\lambda x f(f(x))$
\end{minipage}}

\noindent\fbox{\begin{minipage}{200pt}
  {\bf M-abzählbar} $\exists f.\mathbb{N}\rightarrow M$, Bew: f finden, Diagonalisierung\\
  {\bf M-aufzählbar} \ldots abzählbar und $f_M$ ist berechenbar, Bew: ist nicht $\mathcal{TR}$, Reduktion, Diagonalisierung\\
  {\bf M-entscheidbar} $TM_M$ hält an, $M$ und $\overline{M}$ ist aufzählbar Bew: Rice, Diagonalisierung\\
  {\bf M-semi-entscheidbar} immer auch aufzählbar\\
  {\boldmath$L\in\mathcal{NP}$}  OTMs Lösung beschreiben und Prüfung mit DTM in $\mathcal{P}$ beschr.\\
  {\boldmath$L\in\mathcal{NPC}$}  NPC Prob $L'$ wählen und $x\in L' \Leftrightarrow f(x)\in L \equiv L'\leq_P L$
\end{minipage}}

\noindent\fbox{\begin{minipage}{200pt}
  {\bf Entscheidbar} Abgeschlossen: $\overline{M}, f^{-1}, \cup, \cap, M\setminus N$\\
  {\bf Aufzählbar}  Abgeschlossen: $f^{-1}, f(M), \cup, \cap$
\end{minipage}}

\noindent\fbox{\begin{minipage}{200pt}
  {\bf Partiell-Characteristisch}  
  \[ \psi_M(i)=\left\{ 
    \begin{array}{l l}
      1 & \text{if } i\in M\\
      \perp & \text{sonst}
    \end{array}\right.
  \]
  {\bf Voll-Characteristisch}  $\chi$, wie $\Psi$ nur mit 0 statt $\perp$\\
  {\bf $\varphi_i$} Funktion der $TM_i$\\
  {\bf $\Phi_i$} Schrittfunktion der $TM_i$
\end{minipage}}

\noindent\fbox{\begin{minipage}{270pt}
  Primitive Rekursionsrechnung
  \begin{math}
    h = Pr[c^0_1,g],\; g=Pr[pr^1_1,pr^3_3]\\
    h(0) = 1\\
    h(1) = g(0,h(0)) = g(0,1) = Pr^3_3(0,0,g(0,0)) = g(0,0) = 0\\
    h(2) = g(1,h(1)) = g(1,0) = 1\\
    h(3) = g(2,h(2)) = g(2,1) = Pr^3_3(2,0,g(2,0)) = g(2,0) = 2
  \end{math}
\end{minipage}}

\noindent\fbox{\begin{minipage}{270pt}
  Primitive Rekursion aufstellen
  \begin{math}
    t_=(x,y) \left\{ 
      \begin{array}{l l}
	0 & \text{ falls } x=y\\
	1 & \text{ sonst }
      \end{array}\right.\\
    t_= = sign(add(sub(x,y),sub(y,x)))\\
    t_= = Pr[sign\circ pr^1_1, sign\circ add\circ(sub\circ(pr^3_1,s\circ pr^3_2),sub\circ(s\circ pr^3_2,pr^3_1))]
  \end{math}
\end{minipage}}

\noindent\fbox{\begin{minipage}{270pt}
  $\lambda$-Kalkül $\beta$-Reduktion für
  \begin{math}
    f(x) = 2x + 1\\
    add_f \equiv \lambda n .\lambda f.\lambda x.\overline{1} f \left( ( \lambda f.\lambda x.\overline{2} (n f) x ) f x \right)\\
    f(1) = add_f(\overline{1}) = \lambda n.\lambda f.\lambda x.\overline{1} f \big(\dots\big) \mbox{\boldmath$\overline{1}$}
  \end{math}\\
  Jetzt fein einsetzen und reduzieren\dots\\
  Dann die Church-Numerals durch ihre $\lambda$ Ausdrücke ersetzen\dots\\
  Dann weiter reduzieren\dots\\
  Dann muss sowas rauskommen: $\rightarrow \lambda f.\lambda x.f(f(f(x))) \equiv \lambda f.\lambda x.f^3 (x)\equiv\overline{3}$
\end{minipage}}

\noindent\fbox{\begin{minipage}{200pt}
\end{minipage}}

\end{document}

