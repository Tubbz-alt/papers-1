\documentclass{llncs}
\usepackage{makeidx}  % allows for indexgeneration
\usepackage[pdftex]{graphicx} % PNGs
\usepackage{amsmath, amssymb} % algebra
\usepackage[utf8x]{inputenc}
\usepackage[T1]{fontenc} 
\usepackage[procnames]{listings} % for sourcecode
\usepackage{graphviz} % graphs
\usepackage{array,multirow} % tables
\usepackage{afterpage} % figures
\usepackage{float} % figures

\lstset{%
	basicstyle=\small\ttfamily,
	language=Ruby,
	frame=lines,
	numbers=left,
	numberstyle=\rmfamily\tiny,
	numbersep=3pt,
	breaklines=true,
	breakatwhitespace=true
}

\restylefloat{figure}
\begin{document}
\pagestyle{headings}  % switches on printing of running heads
\mainmatter % start of the contributions
\title{Bithug - Coding Social}
\subtitle{A Code Repository Management Platform and Social Network}
\titlerunning{Bithug}  % abbreviated title (for running head)
\author{Tim Felgentreff~ Konstantin Haase~ Johannes Wollert}
\date{\today}
\authorrunning{Felgentreff, Haase, Wollert}   % abbreviated author list (for running head)
\tocauthor{Tim Felgentreff (Hasso-Plattner-Institute)\\
	   Konstantin Haase (Hasso-Plattner-Institut)\\
	   Johannes Wollert (Hasso-Plattner-Institut)}
\institute{Social Web Applications Engineering, Internet Technologies and Systems, Hasso-Plattner-Institut, Universität Potsdam, D-14482 Potsdam, Germany,\\
\email{\{tim.felgentreff, konstantin.haase, johannes.wollert\}@student.hpi.uni-potsdam.de}}

\maketitle
\begin{abstract}
  This paper describes some of the design decisions taken in creating
  the \emph{Bithug Coding Social} web service. Bithug is a free, highly 
  configurable and social code hosting service created during the ``Social 
  Web Applications Engineering'' seminar at HPI in 2009/10. Its purpose is to 
  provide a free and more versatile alternative to the git hosting 
  site \emph{Github} for universities and companies internal code hosting needs.
\end{abstract}
\section{Introduction}
The increased interactivity of the Web 2.0 has in recent years brought about a 
shift in web technologies. One aspect of this shift is the increasing generation
of content by the consumer, clearly recognizable in the popularity of social 
networks.

In 2007, the classic generators of content on the web, the programmers, got 
their own social network where they could show off their interests and skills: 
the social coding site \emph{Github}\cite{github:www}. On Github, a user can 
create code repositories and freely share code with everyone else on the network.
Other users or projects can be ``followed'' which means, that every action on 
and by the followed user is reported in a feed (on page and via RSS). Commits
can be commented on, people can collaborate on projects and code is online and 
browsable.
All repositories on Github are initially open, however, for a monthly fee you 
can buy private space on Github where only people who have been explicitly 
granted access can view code.
\subsection{Why Social?}
In the past, the sufficiency of self-organizing social networks even on large 
scale projects has been demonstrated by open-source like the Linux 
Kernel\cite{kernel:www}, X.org\cite{xorg:www} or the GNU Project\cite{gnu:www}.
In their footsteps, emerging social websites time\cite{facebook:help} and 
time\cite{facebook:organize} again\cite{twitter:organize} have proven how losely
tied bodies are capable of organizing large events.

In this manner, we want Bithug not only to be a service to host code, but a 
network for exchanging ideas and organizing projects in a smaller university 
or corporate environment to form a ``community'' much like the open-source 
community which has formed on Github.
\subsection{The Lazy Web Concept}
Another change in how the web is used has come about through pages like 
FriendFeed, Twitter and most recently Facebook Lite. They provide short 
messaging services to post notes. \emph{Friends} or \emph{Followers} can
receive and read those messages \ldots or not. \\
This is the concept we the ``Lazy Web'', where content is not generated 
necessarily to be read, but to be there, if anybody is interested. Combined 
with social networks, important or interesting information will find its way 
around as others will carry on the message.

We want Bithug to integrate with services like Twitter to enable our users to
generate content about their activity and projects and possibly discuss them on
other services as well. Also, this way we provide integration with the 
information flow some consumers of the web have come to enjoy.
\section{Project Management as a Service}

DVCS/Git, Our plans, alternatives (Gitorious, GitHub FI, Gitauth)

\section{Technical Stuff}

\subsection{The Stack}
Our project split into three major subprojects: Bithug (out actual project), BigBand (parts of Bithug you can use in any Sinatra Project), MonkeyLib (parts you can use in any Ruby project).

Patched versions of krb5-auth (currently unmaintained) and Ohm.

Other libraries we use: ...

Stack is known to work on ...

\subsection{Persistence}
\subsubsection{Why no RDBS?}
Traditionally, databases have focused on storing relational datasets. 
Relational storage has the advantage, that data retrieval can be formulated in 
a declarative way. These declaration can then be mathematically optimized for 
to meet specific time/space-constraints.
Though relational datasets do not lend themselves very well to object-oriented 
web-programming developers have gone to great length to provide easy-to-use 
object-relational-mappers (ORM) to support RDBS storage in object-oriented 
applications. Popular patterns in ORMs are the ActiveRecord-Pattern
and DataMapper\cite{paikens-use}.

There are several problems with this
\begin{enumerate}
  \item Going through an object-oriented language and mapping objects to 
    relations will usually lead to many smaller queries instead of large 
    ones, leaving less room for the RDBS to optimize. 
  \item Mapping the dynamics of object-inheritance requires stunts like 
    STI/MTI which either store more data than needed or are less efficient in 
    their data-retrieval methods needing large joins to re-create objects.
  \item Dynamic objects with polymorphic attributes require join- and 
    attribute-tables which further reduce the performance of the RDBS.
  \item Populated databases can be very difficult to migrate when the software 
    evolves further away from the premier data-model.
\end{enumerate}

The rich configurability we were aiming for with Bithug would require us to 
provide migration strategies for setups changing their configurations as well 
keeping the mapping general enough to allow easy transitions when implementing 
new services.
\subsubsection{A promising alternative: Maglev}
Maglev is a Ruby implementation atop GemStone/S, the Smalltalk 
object-persistence system. GemStone is an object-oriented database able to 
serialize objects in the runtime environment transparently, taking care of 
object-references and garbage-collection automatically.

GemStone is not a new system\cite{butterworth1991gemstone} for persistence, but only
in recent years has it become sufficiently advanced to compete with traditional
RDMS speed-wise. Its great strength is the built-in support for 
object-orientation: anything the language can do, the server can persist. Using
Maglev would have solved the problem of persistence. However, Maglev is a very 
young implementation of Ruby and it does not yet support enough of the features
we require in Bithug, most importantly the CSS-compiled style-sheet language 
SASS and the Kerberos authentication plug-in we provide for company networks.
\subsubsection{Key-Value-Stores}
Key-value stores have emerged in the last few years as an alternative to 
relational database management systems for some data storage tasks. 
Redis, BigTable, and Casandra are just three of the players in this buzzing
area of database systems. Key-value stores promise to solve the problems we 
met while writing Bithug:
\begin{enumerate}
  \item Fast for a great many small queries
  \item Moderate hardware requirements for small setups
  \item Easy scalability through sharding
  \item Atomic operations and automatic consistency
  \item Elastic, yet simple data model which is easily extensible at runtime
\end{enumerate}

At the most basic level, key-value stores are really simply what the name 
suggests, that is, simple hash-maps or trees which map keys to values. Those 
structures have very fast lookup times for simple key and range queries though 
they get considerably slower and memory hungry when larger and more diverse 
queries are involved. However, if the main task of a persistence layer will be
retrieving values for keys or ranges of keys, these structures \emph{will} 
be\cite{chang2006bigtable} considerable faster than RDBS.

KV-stores like CouchDB or MongoDB provide REST-interfaces to their 
databases. This allows them to be easily replicated behind a load-balancer 
leading to virtually unlimited scalability. Additionally, the data-structures 
and their applications for storage are limited, but simple 
and do not require difficult setup or up-to-date hardware. 

For a web-application starting with
moderate budget and growing over time this is the optimal scenario where the 
return-on-investment can literally be met by the first customer paying for one 
server.

\subsubsection{The Store of our Choice}
In the Ruby world, there are lots of alternative KV-Stores to chose from, some of which 
are fast and simple to use, others which have a more advanced API but provide 
additional features that go beyond a simple hashing KV-Store. Here are a few 
examples of databases we considered using for Bithug.
\begin{enumerate}
  \item
    Tokyo Cabinet, a quite speedy\cite{tcbench} and promising candidate, 
    is a powerful and, being tested and tested again, production-ready tool.
    Despite these advantages, we have made a decision against it, mainly because 
    the original Ruby interface is not pretty to look at at all and the loss of 
    speed when using wrappers such as the rufus-tokyo gem is considerable. 
    Yet, there is some development being made and Tokyo Cabinet might be a 
    Store worth looking into in the future.
  \item
    MongoDB is a document-oriented database, meaning its data is composed of simple 
    key-value constructs that together form a larger chunk of data. It is quite 
    easy to use and provides much of the querying capabilities of the traditional 
    SQL database, though it's queries are not plain as Redis'. Though it is 
    reasonably fast, MongoDB cannot compete with the blazing speed of simpler 
    databases.
  \item
    Cassandra is a quite young Open Source implementation of the BigTable DBS\cite{chang2006bigtable}, 
    with the purpose of creating an easily scalable DBMS. The key to this is 
    the concept of weak consistency, similar to a DNS-Service or 
    Amazons KV-Store Dynamo\cite{decandia2007dynamo}.
    Although it is a promising concept, the Ruby API for Cassandra is neither stable 
    nor attractive, and therefor far from production-ready. Additionally, the 
    BigTable structure is a column-based architecture and needs some time to get 
    used to.
  \item
    Redis is a very lightweight and incredibly fast\cite{haines2009redis} 
    in-memory database. Its Ruby API is very simple, yet effective and 
    sufficiently deep, as it offers easy manipulation of lists and sets. 
    Although Redis does not provide advanced scalability features as Cassandra 
    does, the ruby library supports hashing and distributing over multiple 
    servers. To achieve its remarkable speed however, Redis uses fully 
    asynchronous write operations, thereby sacrificing consistency for speed 
    and scalability.
\end{enumerate}

Because of the ease of use and its raw speed, in addition to the existing 
capabilities of the Ohm persistence model, we decided to use the Redis 
Key-Value Store. However, with the growing usability of backend unaware Ruby 
interfaces such as Moneta, this might not be a final decision.


\subsection{Sinatra}
Sinatra is a ruby web framework\cite{sinatra:www}, lately becoming a popular alternative to Ruby On Rails and Merb.
In contrast to those is has a small code base, does not ship with and persistence layer and does not focus on code
generation. In many cases this will result in better performance than a Rails application, especially for single
purpose applications, since the code size of Rails alone causes such a long dispatch that Sinatra performs much better.

Having a small and clean code base can also be useful, as it is easy for a developer to understand what is going on under the hood.
Also, for some not offering a out of the box ORM solution is a feature, rather than a short-coming, as it is easier to
choose another solution if the system does not assume it is coping with its own ORM (ActiveRecord, in that case). However,
it should be mentioned that those disadvantages have been reduced or removed in the upcoming Rails milestone 3.0.

\subsection{Using dynamic inheritance as means of configuration}
In class-based object-oriented programming inheritance is often used as specialization.
For instance, in an application managing costumers, the class Costumer might have the same
superclass as the class Administrator, as they might share some common logic and attributes.

This behavior can be used for application configuration, where one configuration option can be seen
as a special class inheriting from a more general Application class. In our application we use this
approach for our two core classes: User and Repository.

For instance: You want to use Kerberos authentication. With the previous explanation it could be possible
to have a Kerberos::User class, inheriting from Bithug::User, overwriting the authentication method.
This is actually very close to what we do internally. As you might suspect this approach fails when offering
combinable options. What if you to offer Kerberos and LDAP authentication both as stand-alone solution or
on as a fallback for the other (which is a typical network setup, in our experience). In a language that offers
multiple inheritance, you could create the classes Kerberos::User and Ldap::User, that both inherit from Bithug::User
and than create the classes KerberosWithLdapFallback::User and LdapWithKerberosFallback::User both inheriting
from Kerberos::User and Ldap::User. Would this language not be able to define classes at runtime, it would even be
more complicated, as you would have to generate all possible combinations at compile time. Ruby however does offer
runtime creation of classes. But it does lack multiple inheritance.

A third approach would be to change a classes inheritance chain by altering its superclass at runtime (or at compile
time, for that matter, which would be less dynamic). Upfront: Even though this is possible in most Ruby implementations,
it is considered extremely dangerous\footnote{Apart from maybe even seriously breaking your object space, you would have
to clear a couple of caches used by the underlying Ruby implementation to speed up method dispatch.}, and is not used
by Bithug. It should still be explained, as it helps to understand our implementation for not familiar with the Ruby
method dispatch. Let us take the above example: To configure a system that would first try to authenticate against Kereberos
and if that fails try LDAP authentication, you could change the superclass of Kerberos::User to Ldap::User which still
is a subclass of Bithug::User. If you implement a method Bithug::User.authenticate(login, password), that should return
true if authentication succeeds and false otherwise. Now, if Bithug::User.authenticate always returns false and both
Kerberos::User.authenticate and Ldap::User.authenticate return true if the authentication against LDAP/Kerberos succeeds,
the result of their superclass's authenticate the setup would be complete. This approach is somewhat comparable to context-
or aspect-oriented programming, where you are able to wrap aspects around an object\cite{apel2006aspectual}.

Ruby supports a concept called Mixins\cite{apel2004using}. Mixins are one use case for Ruby modules\footnote{Others are namespacing
and classes without instances.}. A Ruby module is defined like a Ruby class. You can define both instance and singleton
methods\footnote{Methods defined on class side, also known as class methods.}. However, as you cannot instantiate a module,
its instance methods are not directly usable. You can include such a module in a class. It is a common misbelieve – even among
long time rubyists – that by doing so the unbound methods\footnote{Ruby term for not yet belonging to an object, thus not being
callable.} are copied to the class and by doing so overwriting existing methods. In reality when including a module in a class,
a new class is created, containing all the modules instance methods. That class is inserted in the inheritance chain in-between
the original class and its superclass (or previously included modules). This allows a similar usage as changing the superclass
without its complications.

However, if you followed the above explanation closely, you might already see two major issues with that approach. As mentioned,
only the instance methods become part of the new class. The singleton methods are already bound to the module and cannot be
rebound to the class. The solution is a common pattern one will often fine in ruby programs: Use another mixin for the class methods
and include that mixin in the singleton class\footnote{A class every object in ruby has. It keeps all the singleton methods of an
object as instance methods and has that object as sole instance – hence singleton class.}.

The other problem is, that a module is inserted after the class in the inheritance chain, not in front of it. In the Kerberos/LDAP
example, Bithug::User.authenticate would always return false, since the Kerberos and LDAP implementations never get called. Our
solution to that is to have an empty class (i.e. without method definitions) called Bithug::User subclassing Bithug::AbstractUser.
All our common logic is placed inside AbstractUser. Now, if we include a module in Bithug::User, it is inserted in front of AbstractUser
in the inheritance chain, thus getting called.

\section*{Acknowledgments}
\bibliographystyle{splncs}
\bibliography{bithug}
\clearpage
\end{document}
