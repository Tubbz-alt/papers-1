\subsubsection*{Why no RDBS?}
Traditionally, databases have focused on storing relational datasets. 
Relational storage has the advantage, that data retrieval can be formulated in 
a declarative way. These declaration can then be mathematically optimized for 
to meet specific time/space-constraints.
Though relational datasets do not lend themselves very well to object-oriented 
web-programming developers have gone to great length to provide easy-to-use 
object-relational-mappers (ORM) to support RDBS storage in object-oriented 
applications. Popular patterns in ORMs are the ActiveRecord-Pattern
and DataMapper\cite{paikens-use}.

There are several problems with this
\begin{enumerate}
  \item Going through an object-oriented language and mapping objects to 
    relations will usually lead to many smaller queries instead of large 
    ones, leaving less room for the RDBS to optimize. 
  \item Mapping the dynamics of object-inheritance requires stunts like 
    STI/MTI which either store more data than needed or are less efficient in 
    their data-retrieval methods needing large joins to re-create objects.
  \item Dynamic objects with polymorphic attributes require join- and 
    attribute-tables which further reduce the performance of the RDBS.
  \item Populated databases can be very difficult to migrate when the software 
    evolves further away from the premier data-model.
\end{enumerate}

The rich configurability we were aiming for with Bithug would require us to 
provide migration strategies for setups changing their configurations as well 
keeping the mapping general enough to allow easy transitions when implementing 
new services.
\subsubsection*{A promising alternative: Maglev}
Maglev is a Ruby implementation atop GemStone/S, the Smalltalk 
object-persistence system. GemStone is an object-oriented database able to 
serialize objects in the runtime environment transparently, taking care of 
object-references and garbage-collection automatically.

GemStone is not a new system\cite{butterworth1991gemstone} for persistence, but only
in recent years has it become sufficiently advanced to compete with traditional
RDMS speed-wise. Its great strength is the built-in support for 
object-orientation: anything the language can do, the server can persist. Using
Maglev would have solved the problem of persistence. However, Maglev is a very 
young implementation of Ruby and it does not yet support enough of the features
we require in Bithug, most importantly the CSS-compiled style-sheet language 
SASS and the Kerberos authentication plug-in we provide for company networks.
\subsubsection*{Key-Value-Stores}
Key-value stores have emerged in the last few years as an alternative to 
relational database management systems for some data storage tasks. 
Redis, BigTable, and Casandra are just three of the players in this buzzing
area of database systems. Key-value stores promise to solve the problems we 
met while writing Bithug:
\begin{enumerate}
  \item Fast for a great many small queries
  \item Moderate hardware requirements for small setups
  \item Easy scalability through sharding
  \item Atomic operations and automatic consistency
  \item Elastic, yet simple data model which is easily extensible at runtime
\end{enumerate}

At the most basic level, key-value stores are really simply what the name 
suggests, that is, simple hash-maps or trees which map keys to values. Those 
structures have very fast lookup times for simple key and range queries though 
they get considerably slower and memory hungry when larger and more diverse 
queries are involved. However, if the main task of a persistence layer will be
retrieving values for keys or ranges of keys, these structures \emph{will} 
be\cite{chang2006bigtable} considerable faster than RDBS.

KV-stores like CouchDB or MongoDB provide REST-interfaces to their 
databases. This allows them to be easily replicated behind a load-balancer 
leading to virtually unlimited scalability. Additionally, the data-structures 
and their applications for storage are limited, but simple 
and do not require difficult setup or up-to-date hardware. 

For a web-application starting with
moderate budget and growing over time this is the optimal scenario where the 
return-on-investment can literally be met by the first customer paying for one 
server.

\subsubsection*{The Store of our Choice}
In the Ruby world, there are lots of alternative KV-Stores to chose from, some of which 
are fast and simple to use, others which have a more advanced API but provide 
additional features that go beyond a simple hashing KV-Store. Here are a few 
examples of databases we considered using for Bithug.
\begin{enumerate}
  \item
    Tokyo Cabinet, a quite speedy\cite{tcbench} and promising candidate, 
    is a powerful and, being tested and tested again, production-ready tool.
    Despite these advantages, we have made a decision against it, mainly because 
    the original Ruby interface is not pretty to look at at all and the loss of 
    speed when using wrappers such as the rufus-tokyo gem is considerable. 
    Yet, there is some development being made and Tokyo Cabinet might be a 
    Store worth looking into in the future.
  \item
    MongoDB is a document-oriented database, meaning its data is composed of simple 
    key-value constructs that together form a larger chunk of data. It is quite 
    easy to use and provides much of the querying capabilities of the traditional 
    SQL database, though it's queries are not plain as Redis'. Though it is 
    reasonably fast, MongoDB cannot compete with the blazing speed of simpler 
    databases.
  \item
    Cassandra is a quite young Open Source implementation of the BigTable DBS\cite{chang2006bigtable}, 
    with the purpose of creating an easily scalable DBMS. The key to this is 
    the concept of weak consistency, similar to a DNS-Service or 
    Amazons KV-Store Dynamo\cite{decandia2007dynamo}.
    Although it is a promising concept, the Ruby API for Cassandra is neither stable 
    nor attractive, and therefor far from production-ready. Additionally, the 
    BigTable structure is a column-based architecture and needs some time to get 
    used to.
  \item
    Redis is a very lightweight and incredibly fast\cite{haines2009redis} 
    in-memory database. Its Ruby API is very simple, yet effective and 
    sufficiently deep, as it offers easy manipulation of lists and sets. 
    Although Redis does not provide advanced scalability features as Cassandra 
    does, the ruby library supports hashing and distributing over multiple 
    servers. To achieve its remarkable speed however, Redis uses fully 
    asynchronous write operations, thereby sacrificing consistency for speed 
    and scalability.
\end{enumerate}

Because of the ease of use and its raw speed, in addition to the existing 
capabilities of the Ohm persistence model, we decided to use the Redis 
Key-Value Store. However, with the growing usability of backend unaware Ruby 
interfaces such as Moneta, this might not be a final decision.
