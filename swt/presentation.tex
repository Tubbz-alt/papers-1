% sample.tex
\documentclass[pdf,swa,slideBW,nocolorBG,nototal]{prosper}
\usepackage{ngerman}
\usepackage[T1]{fontenc}
%\usepackage[dvips]{geometry}
%\usepackage{arev}
%\usepackage{fancyhdr}
%\usepackage[utf8]{inputenc}

\title{Accumulator}
\subtitle{More Than a Simple Feed-Reader}
\author{Xylander, Wollert, Stoff, Haase, Felgentreff}
%\email{\{thomas.stoff, oliver.xylander, konstantin.haase, tim.felgentreff, johannes.wollert\}@student.hpi.uni-potsdam.de}
\institution{
	Softwaretechnik I \\
        Hasso-Plattner-Institut \\
        Universit�t Potsdam\\
	SS 2009 
}
\foot{Accumulator | Softwaretechnik I 2009 | Felgentreff, Haase, Stoff, Wollert, Xylander | \today }

\begin{document}

\maketitle

\begin{slide}{Introduction}
  \begin{itemize}
  \item Common Feed-Readers have shortcomings
    \begin{itemize}
    \item RSS-Feeds are often broken
    \item Readers have limited input options
    \item Feeds are only presented as they are
    \end{itemize}
  \item The Accumulator was meant to solve some of these
    \begin{itemize}
    \item Filters can fix some feed's problems
    \item We can read more than the common feed formats
    \item Feeds can be tagged and shared
    \end{itemize}
  \end{itemize}
\end{slide}
\begin{slide}{Demo}
  nice picture
\end{slide}
\begin{slide}{Agenda}
  \begin{itemize}
  \item Goals 
  \item XP
  \item System
  \end{itemize}
\end{slide}
\begin{slide}{Goals}
  \begin{itemize}
    \item Read from different sources (RSS, Atom, IMAP, Monticello, Git, \ldots)
    \item Filter sources to ease focus on the things that matter
    \item Be extensible: allow easy integration of new sources and filters
    \item Be open: do not force users into the web-frontend, offer RSS output
  \end{itemize}
\end{slide}
\begin{slide}{Development Process}
  \begin{itemize}
    \item Xtreme Programming
      \begin{itemize}
	\item User stories (**---)
	\item Pair programming (*****)
	\item Stand-up meetings (****-)
	\item TDD (***--)
	\item Continouus integration (****-)
      \end{itemize}
  \end{itemize}
\end{slide}
\begin{slide}{Development Process (contd.)}
  \begin{itemize}
    \item Pair Programming
      \begin{itemize}
	\item Working in two's or three's
	\item Two or three teams working alongside on different parts of the system
	\item Working in one room
	\item Development by smoke-break
      \end{itemize}
  \end{itemize}
\end{slide}
\begin{slide}{Development Process (contd.)}
  \begin{itemize}
    \item Stand-Up Meetings
      \begin{itemize}
	\item Once every week
	\item Bring everyone up-to-date with the latest changes
	\item Discussion of last-weeks changes
	\item Assignment of user stories to new pairs
      \end{itemize}
  \end{itemize}
\end{slide}
\begin{slide}{Development Process (contd.)}
  \begin{itemize}
    \item Test-Driven-Development
      \begin{itemize}
	\item The first hint of an API gets a (failing) test-case
	\item Test cases are in flux - they are examples as much as tests
	\item Test cases that belong to a single interface are completed 
	  in one swift session
	\item SUnit was extend with more testing capabilities (Mutations, Call-Counts)
      \end{itemize}
  \end{itemize}
\end{slide}
\begin{slide}{Development Process (contd.)}
  \begin{itemize}
    \item Continouus integration
      \begin{itemize}
	\item Always have a running version
	\item After each change: run the test suite
	\item Each commit should (within reason) be usable
	\item Non-local commits should always add a feature or fix a bug
	\item Breaking commits should be local
      \end{itemize}
  \end{itemize}
\end{slide}
\begin{slide}{System Overview}
  
\end{slide}
\begin{slide}{Working the Code}
  show some code
  \begin{itemize}
    \item MVC
    \item Convention over Config.
    \item Core Classes
    \item Creator-classes
    \item Testing with mutations
  \end{itemize}
\end{slide}
\begin{slide}{Requirement Tracing: Filters}
  Carl-Maria is pretty annoyed with Fefe's RSS-Feed. He would like
  to have it's title properly displayed in the content area, and 
  at the same time shorten the title to only show the beginning of 
  the first line. To do this, he adds two simple filters to the feed.
  \begin{itemize}
    \item Offer a set of simple, well-defined filters
    \item Allow chaining of those filters to offer powerful filtering
    \item Create automagic test-cases as contract for the I/O requirements
    \item Write the filter functionality and make it pass the tests
    \item Expose the filter through the frontend
  \end{itemize}
\end{slide}
\begin{slide}{Requirement Tracing: Server Ressources}
  The client would like to deploy the system on his server. 
  However, due to limited bandwidth and filesystem space, he cannot 
  afford each feed for each user to be downloaded separately.
  \begin{itemize}
    \item Make sure each feed in the system is unique, so saving and 
      downloading per feed is only done once
    \item Users share the same feeds
    \item Per-user feeds are realized by adding feeds to the user's tags
  \end{itemize}
\end{slide}
\begin{slide}{Requirement Tracing: Tagging}
  Johannes feels, directory hierarchies do not account for 
  the diversity of information available in feeds. He'd much 
  rather assign topics to feeds and be able to select feeds by
  topic. If at it, he would also love to be able to read what other 
  people read on this topic.
  \begin{itemize}
    \item Use tagging instead of directories
    \item Feeds can have arbitrary tags - by selecting 
      tags, the user gets a selection of feeds which belong 
      to all selected tags
    \item A user is just a tag himself - when removing the user 
      from the tag selections, tags from other users can be shown
  \end{itemize}
\end{slide}
\begin{slide}{Iterations and Milestones}
  nice pic
\end{slide}
\begin{slide}{Looking back\ldots}
  What we did well
\end{slide}
\begin{slide}{\ldots on using XPForums \ldots}
  What did we use?\\
  What did we miss?
\end{slide}
\begin{slide}{\ldots and looking ahead}
  what we learned and can improve
\end{slide}
\end{document}
